\documentclass{article}
\usepackage[utf8]{inputenc}

\title{Data Mining - Blatt 01}
\author{Thomas, Manuel, Marius}
\date{\today}

\usepackage{natbib}
\usepackage{graphicx}
\usepackage{subfigure}

\usepackage{listings}
\usepackage{color}

%from https://stackoverflow.com/questions/3175105/writing-code-in-latex-document
\definecolor{dkgreen}{rgb}{0,0.6,0}
\definecolor{gray}{rgb}{0.5,0.5,0.5}
\definecolor{mauve}{rgb}{0.58,0,0.82}

\lstset{frame=tb,
	language=Java,
	aboveskip=3mm,
	belowskip=3mm,
	showstringspaces=false,
	columns=flexible,
	basicstyle={\small\ttfamily},
	numbers=none,
	numberstyle=\tiny\color{gray},
	keywordstyle=\color{blue},
	commentstyle=\color{dkgreen},
	stringstyle=\color{mauve},
	breaklines=true,
	breakatwhitespace=true,
	tabsize=3
}


\begin{document}

\maketitle

\section{Nr 1}
Je höher der Support-Threshold, je weniger Kombinationen in der Ausgabe. Das liegt daran, dass der Support-Threshold angibt, wie hoch die Wahrscheinlichkeit einer Kombination sein muss.

\section{Nr 2}
Histogramme sind im Ordner Histogramme\_Nr2.

\section{Nr 3}
Die Itemsets von dm2 kommen weniger häufig vor als die vom dm3. Bim dm2 gibt es ein 1-Itemsets, welches in 90\% der Fälle vorkommt. dm3 hat relativ viele Kombinationen von Itemsets, deswegen zeigt das Histogramm auch relative viele Häuffungen bei den einzelnen Support-Werten.

\section{Nr 4}

\begin{tabular}{c|c|c}
  Datensatz & Min-Support & Runtime \\
  \hline
  dm1 & 0.4 &  1712 function calls in 0.028 seconds \\
  dm1 & 0.5 &  980 function calls in 0.012 seconds \\
  dm1 & 0.6 &  831 function calls in 0.011 seconds \\
  dm1 & 0.7 &  732 function calls in 0.012 seconds \\
  dm1 & 0.8 &  668 function calls in 0.009 seconds \\
  dm1 & 0.9 &  562 function calls in 0.007 seconds \\
  
  dm2 & 0.4 &  757 function calls in 0.010 seconds \\
  dm2 & 0.5 &  490 function calls in 0.009 seconds \\
  dm2 & 0.6 &  445 function calls in 0.007 seconds \\
  dm2 & 0.7 &  445 function calls in 0.007 seconds \\
  dm2 & 0.8 &  445 function calls in 0.007 seconds \\
  dm2 & 0.9 &  445 function calls in 0.008 seconds \\
  
  dm3 & 0.4 &  26262 function calls in 0.156 seconds \\
  dm3 & 0.5 &  7805 function calls in 0.055 seconds \\
  dm3 & 0.6 &  4143 function calls in 0.030 seconds \\
  dm3 & 0.7 &  2025 function calls in 0.017 seconds \\
  dm3 & 0.8 &  1068 function calls in 0.013 seconds \\
  dm3 & 0.9 &  664 function calls in 0.010 seconds \\
  
  movielines & 0.4 &  1932402 function calls in 13.593 seconds \\
  movielines & 0.5 &  1932391 function calls in 13.791 seconds \\
  movielines & 0.6 &  1932391 function calls in 13.480 seconds \\
  movielines & 0.7 &  1883095 function calls in 13.674 seconds \\
  movielines & 0.8 &  1883084 function calls in 13.977 seconds \\
  movielines & 0.9 &  1883084 function calls in 13.460 seconds \\
\end{tabular}

Je niedriger der Min-Support ist, desto höher wird die Runtime. Das liegt daran, dass der Algorithmus dann viel mehr Möglichkeiten durchgehen muss als wenn der Min-Support größer ist. Bei einem hohem Min-Support schneidet der Algorithmus gleich mehrer Möglichkeiten weg und braucht deswegen weniger Functionsaufrufe und somit eine geringere Runtime.

\section{Nr 5}
Die Ausgabe für dm3 mit einem Min-Support von 0.7 liegt im Ordner abgabe\_nr5. Die csv Datei hat in jeder Zeile ein frequent itemset. Die Labels sind dabei mit einem \",\" getrennt. Die Labels sind die Spalten der Data-CSV.

\end{document}
